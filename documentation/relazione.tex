\documentclass[a4paper,12pt]{article}

\usepackage{latexsym}
\usepackage{amsfonts}
\usepackage{amsmath}
\usepackage{amsthm}
\usepackage{amssymb}

\usepackage{hyperref}
\hypersetup{ colorlinks=true, linkcolor=blue, urlcolor=green,    filecolor=magenta }

\usepackage[utf8]{inputenc}
\usepackage[T1]{fontenc}

\usepackage{textcomp}
\usepackage{lmodern}
\usepackage[a4paper]{geometry} 
\usepackage[italian]{babel}

\newcounter{numtho}
\newtheorem{remark}[numtho]{Remark}
\newcommand{\info}[1]{\texttt{#1}}

\begin{document}

\title{Relazione di progetto\\Lora}
\author{Laurent Claessens}
\maketitle

\tableofcontents

%+++++++++++++++++++++++++++++++++++++++++++++++++++++++++++++++++++++++++++++++++++++++++++++++++++++++++++++++++++++++++++ 
\section{Scopo et descrizione generale del progetto}
%+++++++++++++++++++++++++++++++++++++++++++++++++++++++++++++++++++++++++++++++++++++++++++++++++++++++++++++++++++++++++++

%--------------------------------------------------------------------------------------------------------------------------- 
\subsection{Cosa fa ?}
%---------------------------------------------------------------------------------------------------------------------------

Lora è un programmo che fa due cose.
\begin{itemize}
    \item
        Fa un backup della cartelle \info{\$HOME}
    \item
        Cerca le cartelle che sono \emph{git repository}, verifica se hanno bisogno di un \info{git add}, \info{git commit} oppure di aggiungere dei file dentro \info{.gitignore}. Un interfaccia grafica aiuta a questo tipo di manutenzione.
\end{itemize}

%--------------------------------------------------------------------------------------------------------------------------- 
\subsection{L'algoritmo di backup}
%---------------------------------------------------------------------------------------------------------------------------

Il programma si invoca con\footnote{Vedere il manuale del'utente per maggiore informazioni.}
\begin{center}
      \info{./lora}
\end{center}

In questo esempio, supponiamo fare il backup di \info{<home>} (che sarà usualmente la cartella \info{\$HOME}) dentro la cartella \info{<backup>} (che carà su un disco esterno\footnote{Che dovrebbe essere cifrato, però è un altra storia.}). Di più abbiamo bisogno di una cartelle \info{<purge>} accanto a \info{<backup>}.

Lora fa un \emph{loop} su tutti i file di \info{\$HOME}.

\begin{enumerate}
    \item
        Se \info{<home>/foo/bar.txt} è diverso\footnote{L'attributo \emph{size} o \emph{last write time} è diverso.}  da \info{<backup>/foo/bar.txt}. 
        \begin{itemize}
            \item Sposta \info{<backup>/foo/bar.txt} verso \info{<purge>/foo/bar.txt} Quindi non ci sarà mai una perdita di dati : Lora è più di un programma di sincronizzazione. Lora fa la sincronizzationne di \info{<home>} con \info{<backup>}, ma guarda i viechi files in \info{<purge>}.
            \item Copia \info{<home>/foo/bar.txt} su \info{<backup>/foo/bar.txt}
        \end{itemize}
        Se \info{<home>/foo/bar.txt} ma \info{<backup>/foo/bar.txt} non esista.
        \begin{itemize}
            \item Copia.
        \end{itemize}
\end{enumerate}

Vantaggio di Lora su altri programmi di backup\footnote{Sopratutto quelli che implementano un backup incrementale.} : i dati si ritrovano con una semplice copia di \info{<backup>} si \info{<home>}. L'utente non ha bisogno si Lora per ritrovare i dati. 

In particolare, l'utente può ritrovare i sui dati anche da una chiave USB minimale senza interfaccia grafica. 

%--------------------------------------------------------------------------------------------------------------------------- 
\subsection{L'algoritmo di purge}
%---------------------------------------------------------------------------------------------------------------------------

Quando il backup è finito, Lora fa un \emph{loop} su la cartella \info{<bakcup>}.


Se \info{<backup>/foo/bar.txt} esista ma \info{<home>/foo/bar.txt} non esista, Lora pensa che l'utente abbia eliminato questo file.
\begin{itemize}
    \item 
        Sposta \info{<backup>/foo/bar.txt} verso \info{<purge>/foo/bar.txt}.
\end{itemize}

Nota bene. The purge directory here is not exactly the same as the previous one. In fact each time you launch Lora, a new directory is created :
\begin{center}
    \info{<purge>/<date>/<hour-minutes>}
\end{center}
Inside that directory, Lora creates two directories : \info{modified} and \info{removed} which contain the files that were respectively seen to be modified and removed.

A general concept of Lora is that your data is more precious than your disk space and than everything\footnote{Better to crash than to manage a borderline situation.}. If you understand well the backup/purge concept, you can imagine the extreme disk space waste when you just rename, say your music directory.

%--------------------------------------------------------------------------------------------------------------------------- 
\subsection{Multi-thread}
%---------------------------------------------------------------------------------------------------------------------------

Il backup stesso occupa 2 threads.
\begin{itemize}
    \item Il primo è un \emph{loop} sulla cartella \info{\$HOME} che alla ricerca di files o cartelle di cui aggiogare il backup. Questo thread aggiunge delle \emph{tasks} (vedere \ref{SECooVDGXooHTFdPE}) a una lista.
    \item Il secondo thread legge la lista e esegue le copie da fare.
\end{itemize}
Questi threads non hanno interfaccia grafica, e questo è una scelta di design : devo poter fare un backup anche (e soprattutto) quando il computer va male. Per esempio voglio poter fare il backup da une chiave USB minimale\footnote{Dal punto di vista del codice, il backup è abbastanza ben separato delle dipendenze su Qt. Dovrei scrivere un giorno una versione che si può veramente eseguire fuori del'interfacia grafica.}.

Nello stesso tempo, un elenco dei \emph{git repository} che hanno bisogno di pulizia si aggiorna in un interfaccia grafica.

%--------------------------------------------------------------------------------------------------------------------------- 
\subsection{Perché è cosi grosso ?}
%---------------------------------------------------------------------------------------------------------------------------

Avevo cominciato il lavoro al'inizio di Ottobre 2015 perché avevo bisogno di un programmo di backup per uso personale. La parte di backup era già quasi finita quando abbiamo a parlare del progetto in classe.

Quindi per il progetto stesso, ho aggiunto la parte ``git'' e l'interfaccia grafica.


%+++++++++++++++++++++++++++++++++++++++++++++++++++++++++++++++++++++++++++++++++++++++++++++++++++++++++++++++++++++++++++ 
\section{Gerarchia polimorfa : \info{GenericTask}}
%+++++++++++++++++++++++++++++++++++++++++++++++++++++++++++++++++++++++++++++++++++++++++++++++++++++++++++++++++++++++++++
\label{SECooVDGXooHTFdPE}

%+++++++++++++++++++++++++++++++++++++++++++++++++++++++++++++++++++++++++++++++++++++++++++++++++++++++++++++++++++++++++++ 
\section{Gerarchia polimorfa : \info{MainLoop}}
%+++++++++++++++++++++++++++++++++++++++++++++++++++++++++++++++++++++++++++++++++++++++++++++++++++++++++++++++++++++++++++

\end{document}

 descrizione della gerarchia G: utilità e ruolo dei tipi di G

 descrizione dell’uso di codice polimorfo

 manuale utente della GUI, se l’applicazione lo richiede

