\documentclass[a4paper,12pt]{article}

\usepackage{latexsym}
\usepackage{amsfonts}
\usepackage{amsmath}
\usepackage{amsthm}
\usepackage{amssymb}

\usepackage{hyperref}
\hypersetup{ colorlinks=true, linkcolor=blue, urlcolor=green,    filecolor=magenta }

\usepackage[utf8]{inputenc}
\usepackage[T1]{fontenc}

\usepackage{textcomp}
\usepackage{lmodern}
\usepackage[a4paper]{geometry} 
\usepackage[italian]{babel}


\begin{document}

\title{Relazione di progetto\\Lora}
\author{Laurent Claessens}
\maketitle

\tableofcontents

%+++++++++++++++++++++++++++++++++++++++++++++++++++++++++++++++++++++++++++++++++++++++++++++++++++++++++++++++++++++++++++ 
\section{Scopo et descrizione generale del progetto}
%+++++++++++++++++++++++++++++++++++++++++++++++++++++++++++++++++++++++++++++++++++++++++++++++++++++++++++++++++++++++++++

%--------------------------------------------------------------------------------------------------------------------------- 
\subsection{Cosa fa ?}
%---------------------------------------------------------------------------------------------------------------------------

Lora è un programmo che fa due cose.
\begin{itemize}
    \item
        Fa un backup della cartelle \info{\$HOME}
    \item
        Cerca le cartelle che sono \emph{git repository}, verifica se hanno bisogno di un \info{git add}, \info{git commit} oppure di aggiungere dei file dentro \info{.gitignore}. Un interfaccia grafica aiuta a questo tipo di manutenzione.
\end{itemize}

%--------------------------------------------------------------------------------------------------------------------------- 
\subsection{}
%---------------------------------------------------------------------------------------------------------------------------

<++>

%--------------------------------------------------------------------------------------------------------------------------- 
\subsection{Multi-thread}
%---------------------------------------------------------------------------------------------------------------------------

Il backup stesso occupa 2 threads.
\begin{itemize}
    \item Il primo è un \emph{loop} sulla cartella \info{\$HOME} che alla ricerca di files o cartelle di cui aggiogare il backup. Questo thread aggiunge delle \emph{tasks} (vedere \ref{SECooVDGXooHTFdPE}) a una lista.
    \item Il secondo thread legge la lista e esegue le copie da fare.
\end{itemize}
Questi threads non hanno interfaccia grafica, e questo è una scelta di design : devo poter fare un backup anche (e soprattutto) quando il computer va male. Per esempio voglio poter fare il backup da une chiave USB minimale\footnote{Dal punto di vista del codice, il backup è abbastanza ben separato delle dipendenze su Qt. Dovrei scrivere un giorno una versione che si può veramente eseguire fuori del'interfacia grafica.}.

Nello stesso tempo, un elenco dei \emph{git repository} che hanno bisogno di pulizia si aggiorna in un interfaccia grafica.

%--------------------------------------------------------------------------------------------------------------------------- 
\subsection{Perché è cosi grosso ?}
%---------------------------------------------------------------------------------------------------------------------------

Avevo comminciato il lavoro al'inizio di Ottobre 2015 perché avevo bisogno di un programmo di backup per uso personale. La parte di backup era già quasi finita quando abbiamo a parlare del progetto in classe.

Quindi per il progetto stesso, ho aggiunto la parte ``git'' e l'interfaccia grafica.


%+++++++++++++++++++++++++++++++++++++++++++++++++++++++++++++++++++++++++++++++++++++++++++++++++++++++++++++++++++++++++++ 
\section{Gerarchia polimorfa : \info{GenericTask}}
%+++++++++++++++++++++++++++++++++++++++++++++++++++++++++++++++++++++++++++++++++++++++++++++++++++++++++++++++++++++++++++
\label{SECooVDGXooHTFdPE}

%+++++++++++++++++++++++++++++++++++++++++++++++++++++++++++++++++++++++++++++++++++++++++++++++++++++++++++++++++++++++++++ 
\section{Gerarchia polimorfa : \info{MainLoop}}
%+++++++++++++++++++++++++++++++++++++++++++++++++++++++++++++++++++++++++++++++++++++++++++++++++++++++++++++++++++++++++++

\end{document}

 descrizione della gerarchia G: utilità e ruolo dei tipi di G

 descrizione dell’uso di codice polimorfo

 manuale utente della GUI, se l’applicazione lo richiede

