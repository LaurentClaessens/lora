\documentclass[a4paper,12pt]{article}

\usepackage{latexsym}
\usepackage{amsfonts}
\usepackage{amsmath}
\usepackage{amsthm}
\usepackage{amssymb}

\usepackage{color}
\usepackage{hyperref}
\hypersetup{ colorlinks=true, linkcolor=blue, urlcolor=green,    filecolor=magenta }

\usepackage[utf8]{inputenc}
\usepackage[T1]{fontenc}

\usepackage{listingsutf8}

\usepackage{textcomp}
\usepackage{lmodern}
\usepackage[a4paper,margin=1.5cm]{geometry} 
\usepackage[italian]{babel}



\definecolor{dkgreen}{rgb}{0,0.4,0}
\definecolor{gray}{rgb}{0.5,0.5,0.5}
\definecolor{mauve}{rgb}{0.58,0,0.82}

\lstset{ %
  inputencoding=utf8/latin1,
  backgroundcolor=\color{white},  % choose the background color; you must add \usepackage{color} or \usepackage{xcolor}
  basicstyle=\ttfamily, % \texttt\small,              % the size of the fonts that are used for the code, FIXME \ttfamily
  breakatwhitespace=false,        % sets if automatic breaks should only happen at whitespace
  breaklines=true,                % sets automatic line breaking
  captionpos=b,                   % sets the caption-position to bottom
  commentstyle=\small\color{dkgreen},   % comment style
%  deletekeywords={...},          % if you want to delete keywords from the given language
%  escapeinside={\%*}{*)},        % if you want to add LaTeX within your code
  frame=single,                   % adds a frame around the code
  keywordstyle=\small\color{blue},      % keyword style
  language=C++,                % the language of the code
  fontadjust=false,
  % if you want to add more keywords to the set
%  morekeywords={define,domain,objects,init,goal,problem,action,parameters,precondition,effect,types,requirements,strips,typing},
  numbers=left,                   % where to put the line-numbers; possible values are (none, left, right)
  numbersep=5pt,                  % how far the line-numbers are from the code
  numberstyle=\tiny\color{gray},  % the style that is used for the line-numbers
  rulecolor=\color{black},        % if not set, the frame-color may be changed on line-breaks within not-black text (e.g. comments (green here))
  showspaces=false,               % show spaces everywhere adding particular underscores; it overrides 'showstringspaces'
  showstringspaces=false,         % underline spaces within strings only
  showtabs=false,                 % show tabs within strings adding particular underscores
  stepnumber=1,                   % the step between two line-numbers. If it's 1, each line will be numbered
  stringstyle=\small\color{mauve},      % string literal style
  tabsize=2,                      % sets default tabsize to 2 spaces
  prebreak = \raisebox{0ex}[0ex][0ex]{\ensuremath{\hookleftarrow}}, % pour la fin des lignes.
  aboveskip={1.5\baselineskip},
  %title=\lstname                  % show the filename of files included with \lstinputlisting; also try caption instead of title
%  title=\tiny{File \textcolor{blue}{\url{\lstname}}}          % show the filename of files included with \lstinputlisting; also try caption instead of title
  %% FIXME title !
}
\newcounter{numtho}
\newtheorem{remark}[numtho]{Remark}
\newcommand{\info}[1]{\texttt{#1}}

\begin{document}

\title{Relazione di progetto\\Lora}
\author{Laurent Claessens}
\maketitle

\tableofcontents

%+++++++++++++++++++++++++++++++++++++++++++++++++++++++++++++++++++++++++++++++++++++++++++++++++++++++++++++++++++++++++++ 
\section{Scopo et descrizione generale del progetto}
%+++++++++++++++++++++++++++++++++++++++++++++++++++++++++++++++++++++++++++++++++++++++++++++++++++++++++++++++++++++++++++

%--------------------------------------------------------------------------------------------------------------------------- 
\subsection{Cosa fa ?}
%---------------------------------------------------------------------------------------------------------------------------

Lora è un programmo che fa due cose.
\begin{itemize}
    \item
        Fa un backup della cartelle \info{\$HOME}
    \item
        Cerca le cartelle che sono \emph{git repository}, verifica se hanno bisogno di un \info{git add}, \info{git commit} oppure di aggiungere dei file dentro \info{.gitignore}. Un interfaccia grafica aiuta a questo tipo di manutenzione.
\end{itemize}

%--------------------------------------------------------------------------------------------------------------------------- 
\subsection{L'algoritmo di backup}
%---------------------------------------------------------------------------------------------------------------------------

Il programma si invoca con\footnote{Vedere il manuale del'utente per maggiore informazioni.}
\begin{center}
      \info{./lora}
\end{center}

In questo esempio, supponiamo fare il backup di \info{<home>} (che sarà usualmente la cartella \info{\$HOME}) dentro la cartella \info{<backup>} (che carà su un disco esterno\footnote{Che dovrebbe essere cifrato, però è un altra storia.}). Di più abbiamo bisogno di una cartelle \info{<purge>} accanto a \info{<backup>}.

Lora fa un \emph{loop} su tutti i file di \info{\$HOME}.

\begin{enumerate}
    \item
        Se \info{<home>/foo/bar.txt} è diverso\footnote{L'attributo \emph{size} o \emph{last write time} è diverso.}  da \info{<backup>/foo/bar.txt}. 
        \begin{itemize}
            \item Sposta \info{<backup>/foo/bar.txt} verso \info{<purge>/foo/bar.txt} Quindi non ci sarà mai una perdita di dati : Lora è più di un programma di sincronizzazione. Lora fa la sincronizzationne di \info{<home>} con \info{<backup>}, ma guarda i viechi files in \info{<purge>}.
            \item Copia \info{<home>/foo/bar.txt} su \info{<backup>/foo/bar.txt}
        \end{itemize}
        Se \info{<home>/foo/bar.txt} ma \info{<backup>/foo/bar.txt} non esista.
        \begin{itemize}
            \item Copia.
        \end{itemize}
\end{enumerate}

Vantaggio di Lora su altri programmi di backup\footnote{Sopratutto quelli che implementano un backup incrementale.} : i dati si ritrovano con una semplice copia di \info{<backup>} si \info{<home>}. L'utente non ha bisogno si Lora per ritrovare i dati. 

In particolare, l'utente può ritrovare i sui dati anche da una chiave USB minimale senza interfaccia grafica. 

%--------------------------------------------------------------------------------------------------------------------------- 
\subsection{L'algoritmo di purge}
%---------------------------------------------------------------------------------------------------------------------------

Quando il backup è finito, Lora fa un \emph{loop} su la cartella \info{<bakcup>}.


Se \info{<backup>/foo/bar.txt} esista ma \info{<home>/foo/bar.txt} non esista, Lora pensa che l'utente abbia eliminato questo file.
\begin{itemize}
    \item 
        Sposta \info{<backup>/foo/bar.txt} verso \info{<purge>/foo/bar.txt}.
\end{itemize}

Nota bene. The purge directory here is not exactly the same as the previous one. In fact each time you launch Lora, a new directory is created :
\begin{center}
    \info{<purge>/<date>/<hour-minutes>}
\end{center}
Inside that directory, Lora creates two directories : \info{modified} and \info{removed} which contain the files that were respectively seen to be modified and removed.

A general concept of Lora is that your data is more precious than your disk space and than everything\footnote{Better to crash than to manage a borderline situation.}. If you understand well the backup/purge concept, you can imagine the extreme disk space waste when you just rename, say your music directory.

%--------------------------------------------------------------------------------------------------------------------------- 
\subsection{I dati prima di tutto}
%---------------------------------------------------------------------------------------------------------------------------

Questo programmo è destinato al ``power user'' : l'utente dovrebbe guardate (un po) coda succede dentro il terminale per tenersi al corrente di cosa succede sul suo disco. Per esempio, vedere \( 2000\) files passare dalla purge significa probabilmente che l'utente abbia soppresso una cartella.

Lora contiene ancora un certo numero di casi in cui va in \emph{crash}. Per esempio quando un file è soppresso dal utente tra il momento in cui è visto come ``da aggiornare dentro \info{<backup>}'' e il momento in cui la copia è effettivamente fatta\footnote{Mi capita spesso con dei file ausiliari di \LaTeX\ quando faccio una compilazione nello stesso tempo di un backup.}.

In questi casi, Lora preferisce lasciare al l'utente la responsabilità di capire cos'ha fatto con i suoi dati piuttosto che prendere un iniziativa. Quindi \emph{crash} piuttosto che gestione.

Ci sono dunque un bel po di eccezione di tipo \info{std::string} che sono sollevate e gestite solo alla fine del'\info{main}, sotto la forma di stampa semplice.

In ogni casi, se succede qualcosa di strano con i dati, l'utente deve essere avvertito. Fina ora, l'utente è avvertito sotto la forma d'un \emph{crash}. C'è però una proposta dentro \info{TODO.txt} di usare il sistema di log (vedere \info{Logging.cpp}) per scrivere i problemi dentro un file e stampare il file alla fine del'esecuzione.

%-------------------------------------------------------------------------------------------------------------------------- 
\subsection{Multi-thread}
%---------------------------------------------------------------------------------------------------------------------------

Il backup stesso occupa 2 threads.
\begin{itemize}
    \item Il primo è un \emph{loop} sulla cartella \info{\$HOME} che alla ricerca di files o cartelle di cui aggiogare il backup. Questo thread aggiunge delle \emph{tasks} (vedere \ref{SECooVDGXooHTFdPE}) a una lista.
    \item Il secondo thread legge la lista e esegue le copie da fare.
\end{itemize}
Questi threads non hanno interfaccia grafica, e questo è una scelta di design : devo poter fare un backup anche (e soprattutto) quando il computer va male. Per esempio voglio poter fare il backup da une chiave USB minimale\footnote{Dal punto di vista del codice, il backup è abbastanza ben separato delle dipendenze su Qt. Dovrei scrivere un giorno una versione che si può veramente eseguire fuori del'interfacia grafica.}.

Nello stesso tempo, un elenco dei \emph{git repository} che hanno bisogno di pulizia si aggiorna in un interfaccia grafica.

Per queste ragione, i processi sono lanciate e chiuse in questo modo :
\lstinputlisting{cpp_snip_1.cpp}

\begin{itemize}
    \item Leggere il file di configurazione.
    \item Creare e mostrare l'interfaccia grafica che elenca i \emph{git repository} (\info{GitListWindow}).
    \item Lanciare il thread che legge l'elenco dei task da fare e esegue le task (\info{task\_runner}). A l'inizio, l'elenco è vuota, ovviamente.
    \item Lanciare il backup e la purge (\info{loops}), quindi iniziare a popolare l'elenco delle task da fare.
    \item Aspettare che tutte le task siano finite (\info{task\_runner.join()}).
    \item Aspettare che l'interfaccia grafica sia chiusa (\info{git\_list\_windows.join()}).
\end{itemize}

%--------------------------------------------------------------------------------------------------------------------------- 
\subsection{Perché è cosi grosso ?}
%---------------------------------------------------------------------------------------------------------------------------

Avevo cominciato il lavoro al'inizio di Ottobre 2015 perché avevo bisogno di un programmo di backup per uso personale. La parte di backup era già quasi finita quando abbiamo a parlare del progetto in classe.

Quindi per il progetto stesso, ho aggiunto la parte ``git'' e l'interfaccia grafica.

Di più, questo è un vero programmo che uso veramente per gestire i miei preziosi dati; quindi preferisco avere qualcosa che abbia le funzionalità utile. Storia interessante : avevo già gatto lo stesso programmo molti anni fa\footnote{Le scelte di design sono basate su dei casi di uso reali.} in Python (senza interfaccia grafica). Per finire il lavoro, gli serviva a volte più di trenta minuti. Lora finisce raramente in più di 6 minuti.

%+++++++++++++++++++++++++++++++++++++++++++++++++++++++++++++++++++++++++++++++++++++++++++++++++++++++++++++++++++++++++++ 
\section{Gerarchia polimorfa : \info{GenericTask}}
%+++++++++++++++++++++++++++++++++++++++++++++++++++++++++++++++++++++++++++++++++++++++++++++++++++++++++++++++++++++++++++
\label{SECooVDGXooHTFdPE}

%+++++++++++++++++++++++++++++++++++++++++++++++++++++++++++++++++++++++++++++++++++++++++++++++++++++++++++++++++++++++++++ 
\section{Gerarchia polimorfa : \info{MainLoop}}
%+++++++++++++++++++++++++++++++++++++++++++++++++++++++++++++++++++++++++++++++++++++++++++++++++++++++++++++++++++++++++++

\end{document}

 descrizione della gerarchia G: utilità e ruolo dei tipi di G

 descrizione dell’uso di codice polimorfo

 manuale utente della GUI, se l’applicazione lo richiede

