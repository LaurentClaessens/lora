\documentclass[a4paper,12pt]{article}

\usepackage{latexsym}
\usepackage{amsfonts}
\usepackage{amsmath}
\usepackage{amsthm}
\usepackage{amssymb}
\usepackage{bbm}
\usepackage{graphicx}
\usepackage[normalem]{ulem}

\usepackage{hyperref}
\hypersetup{colorlinks=true,linkcolor=blue}

\usepackage{textcomp}
\usepackage{lmodern}
\usepackage[a4paper]{geometry}	
\usepackage[utf8]{inputenc}
\usepackage[T1]{fontenc}

\newcommand{\info}[1]{\texttt{#1}}

\begin{document}

\title{Lora's user manual}
\author{Laurent Claessens}
\maketitle

Lora is a backup program and a small Git remainder utility\footnote{Did you committed everything you modified today ?}.

\begin{center}
    {\bf Warning}
\end{center}
This is a free software\footnote{In the sense of the GPL : no warranty, etc.} manipulating your precious data. If you use it, I consider you as a consenting adult.

\tableofcontents

%+++++++++++++++++++++++++++++++++++++++++++++++++++++++++++++++++++++++++++++++++++++++++++++++++++++++++++++++++++++++++++ 
\section{Installation and compilation}
%+++++++++++++++++++++++++++++++++++++++++++++++++++++++++++++++++++++++++++++++++++++++++++++++++++++++++++++++++++++++++++

The first step is to download the whole on \href{ https://github.com/LaurentClaessens/lora  }{ github } with
\begin{center}
    \info{git clone https://github.com/LaurentClaessens/lora}
\end{center}

You compile this documentation with
\begin{center}
    pdflatex manual.tex
\end{center}
A \LaTeX\ distribution is needed\ distribution is needed.

%--------------------------------------------------------------------------------------------------------------------------- 
\subsection{The manual way}
%---------------------------------------------------------------------------------------------------------------------------

\begin{enumerate}
    \item
        Create your \info{lora.cfg} looking at \info{example.cfg} for example and explanations.
    \item
        Modify the path of \info{BOOST\_THREAD\_LIB} in \info{makefile}.
    \item
        Test with
        \begin{center}
            \info{./tests.sh}
        \end{center}
        This should compile everything and launch a small testing program.
    \item
        Launch Lora with
        \begin{center}
            \info{./lora}
        \end{center}
\end{enumerate}

%--------------------------------------------------------------------------------------------------------------------------- 
\subsection{The graphical way}
%---------------------------------------------------------------------------------------------------------------------------

%///////////////////////////////////////////////////////////////////////////////////////////////////////////////////////////
\subsubsection{Compile the installation program}
%///////////////////////////////////////////////////////////////////////////////////////////////////////////////////////////

\begin{enumerate}
    \item
        Compile the installation program :
        \begin{center}
            \info{make installation}
        \end{center}
    \item
        Launch the installation program :
        \begin{center}
            \info{./installation}
        \end{center}
\end{enumerate}

%///////////////////////////////////////////////////////////////////////////////////////////////////////////////////////////
\subsubsection{The backup tab}
%///////////////////////////////////////////////////////////////////////////////////////////////////////////////////////////

Here it is :

\begin{center}
    \includegraphics[width=\linewidth]{backup_tab.png}
\end{center}

\begin{enumerate}
    \item
        The backup directory (A) is the directory in which you want your \info{\$HOME} to be copied. Typically it will be something like
        \begin{center}
            \info{/mnt/backup\_partition/backup.lora}
        \end{center}
        where \info{/mnt/backup\_partition} is the mount point of an encrypted partition on an external drive.
    \item
        The purge directory (B) is the directory in which modified and removed files will be copied (from the backup directory) before to the respectively copied from the home to the backup and removed from the backup directory. The main user case is «I accidentally deleted a file, I perform the backup (thus the file is deleted from the backup) and \emph{then} I note that the file was removed.»  In this case, the removed file (in the version that was available in the backup) can be retrieved in the purge directory.

        Typically it will be something like
        \begin{center}
            \info{/mnt/backup\_partition/purge.lora}
        \end{center}

        Note that there are no automatic process removing the files from the purge directory. The size is then always increasing and you should remove very old subdirectory by hand.

        If you are using a program like Thunderbird, I let you know that it will \href{ http://xkcd.com/725/ }{ literally } produce \href{ https://fr.wikipedia.org/wiki/Gibioctet }{ gibis } of data in the purge directory each day.
    \item
        The excluded directories (C) are what they seem to be : they will be excluded from the backup. You can as example not backup the directory in which you save the \info{iso} files of your preferred Linux distribution. Press the \includegraphics{plus.png} button to add an excluded directory.
\end{enumerate}

%///////////////////////////////////////////////////////////////////////////////////////////////////////////////////////////
\subsubsection{The terminal tab}
%///////////////////////////////////////////////////////////////////////////////////////////////////////////////////////////

You can skip this if you don't use git. 

The ``Terminal tab'' allows you to determine what kind of terminal and editor you prefer\footnote{Serious people use \sout{gedit} vim or \sout{notepad}emacs inside \sout{konsole}\,\sout{command.com}terminology.}.
Here is the tab :

\begin{center}
    \includegraphics[width=\linewidth]{terminal_tab.png}
\end{center}
\begin{enumerate}
    \item
        The line (A) ask you the command line that launch your favorite terminal. This will be \info{konsole}, \info{terminology}, \info{xterm} or something like that. This is used when clicking on ``Open a terminal here''.
    \item
        The line (B) ask you how to launch a process in your terminal. The point is that \info{git diff} has to be launched in a terminal in order to take into account you git preferences\footnote{By the way, you should add \info{quotepath=false} in your \info{.gitconfig} file.}.
    \item
        The line (C) ask you your editor. This is used when clicking on ``Edit .gitignore''. If your favorite editor works in a terminal, you need to ask for opening a new terminal, e.g. \info{konsole -e vim} instead of simply \info{vim}.
\end{enumerate}

%///////////////////////////////////////////////////////////////////////////////////////////////////////////////////////////
\subsubsection{The compilation tab}
%///////////////////////////////////////////////////////////////////////////////////////////////////////////////////////////


Here it is :
\begin{center}
    \includegraphics[width=\linewidth]{compilation_tab.png}
\end{center}

This asks you the path of the file \info{libboost\_thread.so} that is mandatory for the compilation of Lora.

%--------------------------------------------------------------------------------------------------------------------------- 
\subsection{Compilation}
%---------------------------------------------------------------------------------------------------------------------------

You need Boost to be installed somewhere on your system\footnote{On my Ubuntu, I do \info{apt install libboost-all-dev  libboost1.58-all-dev }}. The path to the file \info{libboost\_thread.so} has to be correct in the makefile (edit it if you are unsure).

For compiling everything you simply type
\begin{center}
    \info{make all}
\end{center}

%+++++++++++++++++++++++++++++++++++++++++++++++++++++++++++++++++++++++++++++++++++++++++++++++++++++++++++++++++++++++++++ 
\section{Testing}
%+++++++++++++++++++++++++++++++++++++++++++++++++++++++++++++++++++++++++++++++++++++++++++++++++++++++++++++++++++++++++++

You can test Lora by
\begin{center}
    \info{./tests.sh}
\end{center}
This will test some functionalities (for regressions) and allow you to see a Git window. If something gets wrong, send me an email at \info{laurent.claesens@studenti.unipd.it}

%+++++++++++++++++++++++++++++++++++++++++++++++++++++++++++++++++++++++++++++++++++++++++++++++++++++++++++++++++++++++++++ 
\section{Playing with Lora}
%+++++++++++++++++++++++++++++++++++++++++++++++++++++++++++++++++++++++++++++++++++++++++++++++++++++++++++++++++++++++++++

%--------------------------------------------------------------------------------------------------------------------------- 
\subsection{Backup}
%---------------------------------------------------------------------------------------------------------------------------

Launch
\begin{center}
      \info{./lora --configuration=<configuration\_filename>  <starting\_path>  }
\end{center}

\begin{description}
    \item[Configuration] Default is \info{lora.cfg}. The filename in which Lora has to read your configuration.
    \item[Starting path] Default is \info{\$HOME}. Giving the argument overrides the starting path written in the configuration file. If you want to backup only a part of your home because you only have \( 5\) minutes to shut down your computer.
\end{description}
Most of time you only have to launch \info{./lora} with no arguments.

Lora will loop over your home directory (or \info{starting\_path} if given in command line or in the configuration file) and compare\footnote{Files are different if they size differ or if they \info{last\_write\_time} attribute differ.} each file with the corresponding one in your backup directory. 
\begin{enumerate}
    \item
        If \info{<home>/foo/bar.txt} differs from \info{<backup>/foo/bar.txt}. 
        \begin{itemize}
            \item Move \info{<backup>/foo/bar.txt} to \info{<purge>/foo/bar.txt} Thus you will never loose data\footnote{Never in the sense of the GPL no warranty stuff\ldots}. In this sense, Lora is more than a \emph{synchronizing} program. It synchronizes and keeps the old data.
            \item Copy \info{<home>/foo/bar.txt} to \info{<backup>/foo/bar.txt}
        \end{itemize}
        If \info{<home>/foo/bar.txt} exists and \info{<backup>/foo/bar.txt} does not exist.
        \begin{itemize}
            \item Copy.
        \end{itemize}
\end{enumerate}

%--------------------------------------------------------------------------------------------------------------------------- 
\subsection{Purge}
%---------------------------------------------------------------------------------------------------------------------------

When the backup is finished, Lora will loop over your \info{<backup>} directory.

If \info{<backup>/foo/bar.txt} exists and \info{<home>/foo/bar.txt} does not exist, we deduce that you removed your file.
\begin{itemize}
    \item 
        Move \info{<backup>/foo/bar.txt} to \info{<purge>/foo/bar.txt}.
\end{itemize}
Note. The purge directory here is not exactly the same as the previous one. In fact each time you launch Lora, a new directory is created :
\begin{center}
    \info{<purge>/<date>/<hour-minutes>}
\end{center}
Inside that directory, Lora creates two directories : \info{modified} and \info{removed} which contain the files that were respectively seen to be modified and removed.

A general concept of Lora is that your data is more precious than your disk space and than everything\footnote{Better to crash than to manage a borderline situation.}. If you understand well the backup/purge concept, you can imagine the extreme disk space waste when you just rename, say your music directory.

%--------------------------------------------------------------------------------------------------------------------------- 
\subsection{The Git helper}
%---------------------------------------------------------------------------------------------------------------------------

Since Lora loops over the whole \info{\$HOME}, it also takes time to check for each directory if it is a git repository (has non trivial .git subdirectory) and if this repository is clean\footnote{I personally gitted directories like \info{.config} and \info{.kde}, so I am not always conscient that a git repository have been modified.}.

A list of directories that are not clean git repository (untracked or modified files) is displayed. Clicking on one of them opens a dialog window that helps you to manage the situation.


\begin{description}
    \item[Add <this> to gitignore]
\end{description}
<++>

\end{document}
